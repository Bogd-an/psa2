\documentclass[a4paper]{article}
\input{makros} 

\begin{document} 
    \makrosLab{1}{л}{
        Використання сучасних програмних \\
        засобів для автоматизованого  \\
        проектування систем автоматизації
    }

    \section*{Тема роботи}
    Використання сучасних програмних засобів для автоматизованого проектування систем автоматизації.

    \section*{Мета роботи} 
    Метою моєї роботи було ознайомлення з деякими відомими сучасними САПР і набуття практичних навичок 
    у їх використанні для вирішення типових інженерних завдань, зокрема для розробки електричних 
    принципових схем за допомогою програмного продукту Splan 7.0.
 
    \section*{Варіант 15}
    \begin{figure}[h]
        \centering
        \includegraphics[width=0.6\textwidth]{imgs/LW1.0.png}
        \caption*{Рис 1.1: Завдання на створення елекричних елементів}
    \end{figure} 

    
    \begin{figure}[h]
        \centering
        \includegraphics[width=0.5\textwidth]{imgs/LW1.2.png}
        \caption*{Рис 1.2: Завдання на створення електричної схеми}
    \end{figure} 
    
    \newpage 

    \section*{Хід роботи}
    Першим етапом роботи стало ознайомлення з теоретичними відомостями щодо програмних засобів для 
    автоматизованого проектування (САПР). Я вивчив основні функції програм, таких як Splan 7.0, які 
    дозволяють ефективно розробляти електричні принципові схеми. Важливим аспектом є можливість 
    створювати схеми, що відповідають вимогам державних стандартів.

    Далі я приступив до використання програми Splan 7.0 для створення електричної принципової схеми 
    згідно з моїм індивідуальним завданням. Я вибрав відповідний тип схеми, визначив елементи та їх 
    взаємозв'язки, після чого приступив до нанесення елементів на схему. Кожен елемент я ретельно 
    позначав відповідно до вимог стандартів ЄСКД, що включають точні позначення компонентів електричних 
    схем, що є важливим для правильного розуміння та подальшого використання схеми.

    Після розробки схеми я створив нові елементи бібліотеки, що дозволило мені розширити можливості 
    програми та врахувати специфіку завдання. Створення бібліотеки елементів є важливим етапом, 
    оскільки дозволяє заощадити час при подальшій роботі та дає змогу швидко додавати нові компоненти 
    до проекту.


    \begin{figure}[h]
        \centering
        \includegraphics[width=0.5\textwidth]{imgs/LW1.01.png}
        \caption*{Рис 1.3: Трьох обмотковий трансформатор}
    \end{figure} 

    \newpage
    
    \begin{figure}[h]
        \centering
        \includegraphics[width=0.3\textwidth]{imgs/LW1.02.png}
        \caption*{Рис 1.4: Біполярний транзистор}
    \end{figure} 

    \begin{figure}[h]
        \centering
        \includegraphics[width=0.5\textwidth]{imgs/LW1.03.png}
        \caption*{Рис 1.5: Силовий трьохфазний трансформатор}
    \end{figure} 

    
    \section*{Висновок}
    В результаті виконаної роботи я ознайомився з основними функціями програмного забезпечення 
    для автоматизованого проектування, зокрема Splan 7.0. Я набув практичних навичок у створенні 
    електричних принципових схем та освоїв процес позначення елементів згідно з державними стандартами. 
    Також я створив новий елемент бібліотеки, що допомогло покращити ефективність роботи в програмі. 
    В цілому, робота дозволила мені краще зрозуміти можливості сучасних САПР та їх використання для 
    розв'язання типових інженерних завдань.

    \section*{1.3. Контрольні питання}
    \begin{enumerate}
        \item \textbf{Що забезпечує САПР?} \\
        САПР (система автоматизованого проектування) забезпечує автоматизацію процесів проектування, 
        що включає створення, редагування та аналіз різних видів технічних схем, розрахунків і моделей. 
        Вона значно зменшує час, необхідний для проектних робіт, підвищує точність та якість проектування, 
        а також дозволяє створювати проекти, які відповідають стандартам.

        \item \textbf{Що означає САD?} \\
        САD (Computer-Aided Design) — це система комп'ютерного проектування, яка використовується для 
        створення, зміни, аналізу та оптимізації проектів. САПР є частиною цієї системи і охоплює програмні 
        засоби, які допомагають проектувати різні технічні об'єкти, зокрема електричні схеми.

        \item \textbf{Чому електричні схеми складають за вимогами ЄСКД?} \\
        ЄСКД (Єдина система конструкторської документації) визначає стандарти та вимоги до оформлення 
        технічної документації. Електричні схеми складають за цими вимогами для того, щоб забезпечити 
        уніфікацію позначень елементів, зрозумілість для інших спеціалістів, а також відповідність всім 
        нормативам та стандартам в галузі.

        \item \textbf{Що таке ЄСКД?} \\
        ЄСКД (Єдина система конструкторської документації) — це система, що включає стандарти і норми 
        для розробки та оформлення проектної документації. Вона визначає вимоги до зовнішнього вигляду 
        документів, позначення елементів, а також правила виготовлення креслень та схем, що забезпечують 
        правильне розуміння та використання документів.
    \end{enumerate}

    \begin{figure}[h]
        \centering
        \includegraphics[width=0.8\textwidth]{imgs/LW1.3.png}
        \caption*{Рис 1.6: Електрична схема}
    \end{figure} 

\end{document}