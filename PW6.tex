\documentclass[a4paper]{article}
\usepackage{listings}
\input{makros}

\begin{document}
    \makrosLab{6}{п}{
    Розробка алгоритмів роботи\\
    мікроконтролерів}
\section*{Завдання}
Варіант 15

Розробити алгоритм роботи мікроконтролера керування індикаторною
сигналізацією



\section*{Покрокове пояснення алгоритму}
\begin{enumerate}
    \item \textbf{Старт}: запуск мікроконтролера після подачі живлення або перезапуску.
    \item \textbf{Ініціалізація виводів індикації}: налаштування виводів мікроконтролера, які керують світлодіодами, сиренами тощо.
    \item \textbf{Ініціалізація входів для датчиків}: налаштування пінів для прийому сигналів з датчиків (руху, диму, температури тощо).
    \item \textbf{Ініціалізація таймерів (за потреби)}: налаштування внутрішніх таймерів для контролю часу або затримок.
    \item \textbf{Основний цикл перевірки системи}: постійне опитування датчиків у циклі, поки система увімкнена.
    \item \textbf{Зчитування сигналів з датчиків}: перевірка стану всіх підключених датчиків.
    \item \textbf{Аналіз тривожної ситуації}:
    \begin{itemize}
        \item Якщо виявлено тривогу — активується відповідна індикація.
        \item Подія записується у журнал або передається через комунікаційний модуль.
    \end{itemize}
    \item \textbf{Якщо тривога не виявлена} — індикатори вимикаються або переводяться в режим очікування.
    \item \textbf{Затримка / очікування переривання}: коротка пауза для зменшення навантаження на процесор або очікування подій.
    \item \textbf{Повернення до основного циклу}: перевірка, чи система ще активна, і повторення процесу.
\end{enumerate}


\newpage

\begin{figure}[h]
    \centering
    \includegraphics[width=0.9\textwidth]{imgs/PW6.drawio.png}
    \caption*{Рис: 6.1 Блок-схема алгоритму }
\end{figure} 



\section*{Висновок}
У ході виконання практичної роботи я навчився  розробляти блок-схеми алгоритмів програми для мікроконтроллерів згідно ДСТУ.


\end{document}
\end{document}