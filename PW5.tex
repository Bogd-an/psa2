\documentclass[a4paper]{article}
\usepackage{listings}
\input{makros}

\begin{document}
    \makrosLab{5}{п}{Побудова блок-схем алгоритмів}
\section*{Тема роботи}
Побудова блок-схем алгоритмів

\section*{Мета роботи}
Навчитись розробляти блок-схеми алгоритмів згідно ДСТУ.

\section*{Завдання}
Варіант 15

Скласти блок-схему алгоритму обчислення значень змінної згідно варіанту



\section*{Алгоритм розрахунку функції $Y$}

\[
Y = 
\begin{cases}
\dfrac{\sin^2 \gamma - x^2}{\cos y + \sqrt{x^3 + 1}}, & \text{якщо } x \leq 0 \\
x = \gamma^2 + a^2, & \text{якщо } x > 0
\end{cases}
\]

\noindent
де: 
\begin{align*}
    a &= 10, \\
    \gamma &= 0{,}35 \cdot a.
\end{align*}

\section*{Алгоритм}

\begin{enumerate}
    \item Присвоїти значення: $a := 10$
    \item Обчислити: $\gamma := 0{,}35 \cdot a$
    \item Ввести значення $x$
    \item \textbf{Якщо} $x \leq 0$, тоді:
    \begin{enumerate}
        \item Ввести значення $y$
        \item Обчислити чисельник: $\sin^2(\gamma) - x^2$
        \item Обчислити знаменник: $\cos(y) + \sqrt{x^3 + 1}$
\newpage
        \item Обчислити $Y = \dfrac{\sin^2(\gamma) - x^2}{\cos(y) + \sqrt{x^3 + 1}}$
        \item Вивести значення $Y$
    \end{enumerate}
    \item \textbf{Інакше} (тобто якщо $x > 0$):
    \begin{enumerate}
        \item Обчислити $x = \gamma^2 + a^2$
        \item Вивести значення $x$
    \end{enumerate}
\end{enumerate}

% ```mermaid
% graph TD;
%     A(Початок) --> B{Чи задано x?};
%     B -- Так --> C[Отримати x];
%     B -- Ні --> D[Обчислити x = 1.7 - e^0.35];
%     C --> E{Яке значення x?};
%     D --> E;
%     E -- x ≤ 1 --> F[Y = 0.5 cos(x) + 4x];
%     E -- x < 0 --> G[Y = 0.25x^4 + 2x^2];
%     E -- x > 1 --> H[Y = 0.9√x - 0.8x];
%     F --> I(Вивести Y);
%     G --> I;
%     H --> I;
%     I --> J(Кінець);
% ```

\begin{figure}[h]
    \centering
    \includegraphics[width=0.5\textwidth]{imgs/PW5.drawio.png}
    \caption*{Рис: 5.1 Блок-схема алгоритму рівняння}
\end{figure}

\section*{Висновок}
У ході виконання практичної роботи я навчився  розробляти блок-схеми алгоритмів згідно ДСТУ.


\end{document}