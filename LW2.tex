\documentclass[a4paper]{article}
\input{makros}
\usepackage{longtable}

\begin{document}
    \makrosLab{2}{л}{
        Розробка та складання схем \\
        електричних принципових керування \\ 
        промисловими двигунами
    }

    \section*{Тема роботи}
    Розробка та складання схем електричних принципових керування
    промисловими двигунами

    \section*{Мета роботи}
    Вивчити будову та принцип дії промислових двигунів
    різних типів, як складових систем автоматичного
    керування / регулювання / контролю. Навчитися складати схеми електричні
    принципові для керування промисловими двигунами різних типів.
    
    \section*{Завдання}
Трифазний асинхронний двигун з короткозамкненим ротором має такі параметри:
\begin{enumerate}
    \item напруга живлення: $380/220$ В;
    \item номінальна потужність на валу: $P_{\text{ном.мех}}$;
    \item номінальна швидкість: $n_{\text{ном}}$;
    \item коефіцієнт корисної дії: $\eta$;
    \item коефіцієнт потужності: $\cos \varphi_{\text{ном}}$;
    \item коефіцієнт кратності пускового струму: $\alpha$;
    \item коефіцієнт кратності пускового моменту: $\beta = \frac{M_{\text{пуск}}}{M_{\text{н}}}$;
    \item коефіцієнт перенавантажної здатності: $\gamma = \frac{M_{\text{max}}}{M_{\text{н}}}$.
\end{enumerate}

\newpage

\section*{Вихідні дані}
\begin{itemize}
    \item Потужність на валу: $P_{\text{мех}} = 2{,}8 \text{ кВт}$
    \item ККД: $\eta = 0{,}86$
    \item Косинус фі: $\cos \varphi = 0{,}88$
    \item Напруга: $U = 380 \text{ В}$
    \item Частота: $f = 50 \text{ Гц}$
    \item Частота обертання: $n = 1440 \text{ об/хв}$
    \item Пусковий струм: $\alpha = 5{,}6$
    \item Пусковий момент: $\beta = 2{,}2$
    \item Критичний момент: $\gamma = 2{,}3$
\end{itemize}

\section*{Результати розрахунків}

\subsection*{1. Активна потужність}
\[
P_{\text{ел}} = \frac{P_{\text{мех}}}{\eta} = \frac{2{,}8}{0{,}86} \approx 3{,}256 \text{ кВт}
\]

\subsection*{2. Повна потужність}
\[
S = \frac{P_{\text{ел}}}{\cos \varphi} = \frac{3{,}256}{0{,}88} \approx 3{,}7 \text{ кВА}
\]

\subsection*{3. Реактивна потужність}
\[
Q = \sqrt{S^2 - P_{\text{ел}}^2} = \sqrt{3{,}7^2 - 3{,}256^2} \approx 1{,}708 \text{ кВАр}
\]

\subsection*{4. Лінійний струм}
\[
I = \frac{S \cdot 10^3}{\sqrt{3} \cdot U} = \frac{3700}{\sqrt{3} \cdot 380} \approx 5{,}63 \text{ А}
\]

\subsection*{5. Пусковий струм}
\[
I_{\text{пуск}} = \alpha \cdot I = 5{,}6 \cdot 5{,}63 \approx 31{,}53 \text{ А}
\]

\subsection*{6. Ємність компенсуючих конденсаторів}
\[
Q_{\text{конд}} = P_{\text{ел}} (\tan \varphi_1 - \tan \varphi_2)
\]

\[
\tan \varphi_1 = \tan(\arccos(0{,}88)) \approx 0{,}538,\quad \tan \varphi_2 = \tan(\arccos(0{,}95)) \approx 0{,}328
\]

\[
Q_{\text{конд}} = 3{,}256 \cdot (0{,}538 - 0{,}328) \approx 0{,}684 \text{ кВАр}
\]

\textbf{Для з'єднання «зірка»}:
\[
C_Y = \frac{Q_{\text{конд}} \cdot 10^3}{2 \pi f \cdot 3 U_{\text{ф}}^2} = \frac{684}{2 \pi \cdot 50 \cdot 3 \cdot 220^2} \approx 7{,}55 \, \mu\text{Ф}
\]

\textbf{Для з'єднання «трикутник»}:
\[
C_\Delta = \frac{684}{2 \pi \cdot 50 \cdot 3 \cdot 380^2} \approx 3{,}17 \, \mu\text{Ф}
\]

\subsection*{7. Момент на валу}

\[
\omega = \frac{2\pi n}{60} = \frac{2\pi \cdot 1440}{60} \approx 150{,}8 \, \text{рад/с}
\]
\[
M_{\text{ном}} = \frac{P_{\text{мех}} \cdot 10^3}{\omega} = \frac{2800}{150{,}8} \approx 18{,}57 \, \text{Нм}
\]
\[
M_{\text{пуск}} = \beta \cdot M_{\text{ном}} = 2{,}2 \cdot 18{,}57 \approx 40{,}85 \, \text{Нм}
\]
\[
M_{\text{кр}} = \gamma \cdot M_{\text{ном}} = 2{,}3 \cdot 18{,}57 \approx 42{,}71 \, \text{Нм}
\]

\subsection*{8. Ковзання}
\[
s_{\text{ном}} = \frac{1500 - 1440}{1500} = 0{,}04
\]
\[
s_{\text{кр}} = s_{\text{ном}} \cdot (\gamma + \sqrt{\gamma^2 - 1}) = 0{,}04 \cdot (2{,}3 + \sqrt{2{,}3^2 - 1}) \approx 0{,}179
\]

\subsection*{9. Залежність моменту від ковзання}

\[
M(s) = \frac{M_{\text{кр}}}{\frac{s_{\text{кр}}}{s} + \frac{s}{s_{\text{кр}}}}
\]

\newpage

\begin{longtable}{|c|c|}
\hline
\textbf{Ковзання, $s$} & \textbf{Момент, Нм} \\
\hline
0 & 0 \\
0.04 & 18.57 \\
0.143 & 39.61 \\
0.179 & 42.71 \\
0.215 & 39.23 \\
0.2 & 31.3 \\
0.4 & 35.2 \\
0.6 & 37.6 \\
0.8 & 39.0 \\
1.0 & 39.7 \\
\hline
\end{longtable}


\begin{figure}[h]
    \centering
    \includegraphics[width=1\textwidth]{imgs/LW2.5.png}
    \caption*{Рис. 2.6: Залежність обертаючого моменту $M$ від ковзання $s$}
\end{figure} 


\section*{Висновки}

Отримані результати дозволяють оцінити параметри роботи трифазного асинхронного двигуна, його енергетичні характеристики та вибір необхідних ємностей для підвищення коефіцієнта потужності.

\section*{Контрольні питання}
\begin{enumerate}
    \item Чому асинхронний двигун так називається? \\
    Асинхронний двигун називається так тому, що частота обертання його ротора не співпадає з частотою обертання магнітного поля статора (яка визначається частотою змінного струму). Різниця між цими частотами називається ковзанням.
    
    \item Чому є небажаною велика сила пускового струму? \\
    Велика сила пускового струму небажана, оскільки вона може призвести до значних механічних та електричних навантажень на двигун і мережу, викликати пошкодження ізоляції проводів, зменшити термін служби обладнання, а також викликати перевантаження трансформаторів і підстанцій.
    
    \item Що використовують для зниження сили пускового струму? \\
    Для зниження сили пускового струму використовують спеціальні пристрої, такі як стартери з обмеженням струму, трансформатори з регульованим напругою або пристрої плавного пуску, що забезпечують поступове збільшення напруги на двигуні.
\end{enumerate}


\end{document}